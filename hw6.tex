\documentclass[titlepage]{article}
\usepackage{listings}
\title{CS 180 Homework 6}
\author{Robert Geil \\
University of California, Los Angeles
}
\lstset{frame=tb,
  showstringspaces=false,
  columns=flexible,
  basicstyle={\small\ttfamily},
  breaklines=true,
  tabsize=2
}

\begin{document}
\maketitle

\section{Signal Interleaving}
\subsection{Problem}
A listener is observing a signal arising from two ships. Each signal pattern is a repeating sequence, such that $x$ is a prefix of $x^k$.
Given an output $S$, where $S$ is a bit string, and two patterns $x$ and $y$, determine if $S$ is an interleaving of $x$ and $y$.
\subsection{Algorithm}
The solution for this problem is to use dynamic programming. As we add bits from zero to $S$, each additional bit is either part of a repeat
of $x$ or $y$. In order to solve the case for any given point, the new incoming bit is either a part of the current iteration of x', or the
current iteration of y' or neither
\begin{lstlisting}
1. Create a three dimensional grid with dimension 1 length of the input (1 to |S|), dimension 2 as repetitions of x from 1 to the length of S, and dimension 3 as repetitions of y from 1 to length of S. For example, if x is 101, the values on the x dimension would be 1, 10, 101, 1011, ...
2. For each cell, mark the value false
3. For the first cell (0,0,0), mark it true if the first value of S is either the first value of x or y
4. Proceeding through row-wise, for each position in the grid, if the new position of S matches with x's latest set the grid value to (i, j-1, k) or y's latest value set the grid to (i, j, k-1)
5. Otherwise, set to the value of the (i, j-1, k-1) position.
6. Linearly search through the grid where the size is S, and x and k are variable. If any of these values are true, the signal is an interleaving.
\end{lstlisting}
\subsection{Proof}
In order to prove this algorithm, we examine the inductive step and base case for proceeding through the grid. For the base case, it is
trivially true that if the input is of size 1, and the value of the first input matches either x[0] or y[0], then this is an interleaving, by
the definition of an interleaving presented in the problem. For the inductive step, we again see that at every position, if the most recent to
add value is the "next expected value" for either x or y, by adding it, we maintain the truth of an interleaving given that the previous
value was an interleaving. Finally, for the case where the new value isn't expected for x or y, we look back and recurse to a prior step where
the most recent x and most recent y are removed, and see if this then provides the solution, in such a manner performing an exhaustive search.
\subsection{Runtime}
The runtime of this algorithm is O($n^3$) where n is the length of the input sequence. Because the constructed grid is three dimensional, and every
node is visited in a row-major order, we see that there must be at least $n^3$ calculations performed. 

\section{Number of Shortest Paths}
\subsection{Problem}
Given a weighted graph $G$, where all edges may be positive or negative integer weights, but with the restriction that any cycles have
positive weight, provide an algorithm to find the number of shortest paths between two nodes $u$ and $v$.
\subsection{Algorithm}
While solving the general problem may be difficult, finding the solution in an inductive manner is fairly easy. For the base case, we just
need to find the shortest paths a distance of 1 from $u$ (eg just the neighbors of $u$). For the inductive step, we need to just observe
the distance from all $n-1$ segment nodes and the paths from them to the $n$ nodes.
\begin{lstlisting}
1. Set the distance to all nodes to be infinity, and the distance to u to be zero. Set the number of shortest paths to u to be 1
2. Create a set L0 just u. Set k = 0
3. Create a set L(k+1) with the direct neighbors of L(k)
4. For each element n in L(k+1), find the minimum of L(k)i + ei for all i in L(k) where ei exists as an edge from L(k)i to n. If multiple such minima exist, take the number of paths to be the sum of the number of paths to each minimum
5. If the minimum value is less than the current minimum on n, set the min of n to be that minimum value and the number of paths to be the number calculated in step 4
6. If the minimum is equal to the current minimum on n, add the number of paths calculated in step 4 to n's number of paths
7. While k < the number of nodes in the graph - 1, set k = k + 1, go to step 3
8. The result is the number of minimum paths at node v
\end{lstlisting}
\subsection{Proof}
We will prove the accuracy of this algorithm inductively. We see for the base case of zero segments ($L = 0$), it is trivially true that there
is one path from $u$ to itself. For the $n$th case, we assume that the optimal solution that is $L = n$ is known for all nodes. Therefore,
finding the optimal path of length $n + 1$, we examine all elements that are a distance of 1 away from the set of nodes a distance of $n$ away.
For each of these, the minimum distance is either the current minimum distance, or a distance from one of the neighboring nodes plus the distance
of that neighboring node to $u$. \\
\textbf{Case 1:} The existing distance is smaller than all new distances from nodes $n$ away from $u$. With this case, it is intuitively true
that we have choosen the correct shortest path and the correct number of shortest paths by not updating \\
\textbf{Case 2:} The existing distance is less than one or more of the new distances from nodes $n$ away. With this case, we show that we have
found a shorter distance to the given node, improving our previous minimum distance. In addition, updating the number of paths based on the sum
of the number of paths from the minima is correct. Since the new node may be reached by any of these paths, and each of these paths can be reached
by the number of paths to the previous node, adding them provides the number of paths of the same minimum distance.\\
\textbf{Case 3:} The existing distance is equal to one or more incoming paths. If this is the case, we see that the minimum distance does not
need to be updated, but more paths of the same minimum distance have been discovered. Therefore, to find $L = n+1$ solution, we take the existing
number of paths and add all routes of length $L = n$.
\subsection{Runtime}
Overall the runtime of this algorithm is O($n^3$) where $n$ is the number of nodes in the graph. We see that our outer loop proceeds while
$k$ is less than the number of nodes in the graph. This is because for any graph of $n$ nodes, there cannot be a distance without loops of
more than $n - 1$ segments. Since we assume no negative cycles, proceeding through any cycle will give a longer path, and by cutting the cycle
out, will give a more optimal solution. Within this loop, all neighbors are investigated, and in the worst case there may be $n$ neighbors.
For each of these, the minimum path from a previous value is investigated. Finding the minimum is again an O($n$) operation. Therefore we
see our overall runtime if O($n^3$).

\section{Gerrymandering}
\subsection{Problem}
Given a set of $n$ precincts, each of which has a number of members of party A and party B, determine if the district is susceptible to gerrymandering.
That is, is it possible to divide the precincts into two groups of size $n/2$ where both belong to the same party. Each precinct has equal numbers of
people.
\subsection{Algorithm}
We see that in essence, this is an $n \choose {n/2}$ problem, as we need to select $n/2$ precincts for each district. Since the runtime of this is
O($2^n$), we need to employ dynamic programming to bring it down to polynomial time. We can use some simplifying cases and build up inductively,
such as the fact that the case with precincts of size 1 is trivial, and is the case of a small number (2) precincts.
\begin{lstlisting}
1. Create an NxM grid, where N is the number of precincts, and M is the number of voters in a precinct. Allow N to only be even numbers (2, 4, etc.)
2. For the (2, 1) position, mark true if both of the two voters in the two precincts are of party A, and mark D1 and D2 to be a value of 1 for A and 0 for B
3. Going row-wise with increasing precinct numbers:
4. If both precincts have majority for A and the (n-1, m) cell can be gerrymandered, assign arbitrarily to D1 and D2, and add (n-1, m) to the current values.
5. If both precincts have a majority for A and the (n-1, m) cell cannot be gerrymandered, test the assignment of the two new precincts. If one assignment causes the (n-1, m) cell to be gerrymandered, use that. Otherwise, select the one which brings the lower district more A votes
6. If the precincts are different, assign in such a way that gerrymandering can still be done (ie if A can gerrymander before, assign the two new ones such that this continues. If A could not gerrymander, add the majority A precinct to the minority A district)
7. If both precincts are for party B, assign in such a way that gerrymandering can still be done (as above)
8. Proceed to the increase the number of voters in the district to include one more (ie m = m + 1). Go to step 3
9. If the lower left (N, M) position denotes gerrymandering can be done, return that A can gerrymandered
10. Switch all Bs to A and all As to B, and repeat step 1
\end{lstlisting}
\subsection{Proof}
Because this algorithm relies on an exhaustive search, and examines all possibilities when adding a new precinct, we see that this follows
the dynamic programming principles. Our base case is trivially easy to examine, and can be seen that for two precincts of size 1, gerrymandering can
only be performed if both voters are for the same party. By induction we see that at every node in the grid, there must be an optimal 
solution (whether one party can gerrymander). Therefore, when we examine the final grid square, this too must by induction have a correct 
solution. 
\subsection{Runtime}
Since this solution builds up a grid for 0-N precincts and 0-M voters in a precinct, we see that the overall runtime must be O($MN$). Since
the work done at each step is constant (examining a countable number of cases and performing some lookups for each of those cases), the
time complexity of the algorithm is driven by the size of the grid, in this case $MN$.

\section{Base Station Connections}
\subsection{Problem}
In a given area, individuals want to connect their $n$ clients to $k$ base stations. Each client has an $(x, y)$ coordinate, and each base station
has a range $r$ which is the maximum distance that can be connected to, and a load factor $L$ which is the maximum number of devices that
can be connected to the station. Determine if the network can support all clients.
\subsection{Algorithm}
This problem relies on the Ford-Fulkerson Algorithm, by representing the problem as a graph
\begin{lstlisting}
1. Construct a graph G containing all base stations, and all clients.
2. Add a virtual source S and a virtual sink T to the graph
3. For each base station s, for each client c, if the euclidian distance between s and c is less than the radius r which is the max distance for s, create an edge of weight 1 directed from c to s
4. For each base station s, add a directed edge from s to T with weight L (where L is the capacity of the base station)
5. For each client c, add a directed edge from S to c with infinite weight.
6. Run Ford-Fulkerson algorithm. If the result max-flow = n, the network can handle everyone. Otherwise it cannot.
\end{lstlisting}
\subsection{Proof}
We will prove that our model of the graph is accurate, and that the result will be correct. First, we look at the assigning of links from
clients to base stations. Creating an edge from each client to base station it is in range with gives us an accurate graph, as this guarantees
that there is no connection between any client and a base station that it is out of range with. For the links from each base station to the
virtual sink, by limiting the value as $L$, the capacity of the base station, we guarantee that not more than $L$ can pass through the
base station. By contradiction, if there were more incoming clients than the capacity, since the output flow is $L$, this would mean that
the input and output of the base station would not be equal, giving a contradiction on the flow graph. Finally we see that by proofs done
in class that Ford-Fulkerson will always resolve a max-flow in a graph of integer weights. Since all connected clients generate a flow of
$n$ if they are all connected (as each client produces 1 flow), if Ford-Fulkerson finds a max-flow of $n$, then the network can handle everyone.
Conversely, if the result of Ford-Fulkerson is less than $n$, then there cannot be a flow which generates more than the result of Ford-Fulkerson,
and therefore the network cannot be connected.
\subsection{Runtime}
The overall runtime of this algorithm is the runtime to create the graph, plus the runtime to perform Ford-Fulkerson. In order to generate the
graph, we need to find the distance from each client to each base station. Finding the distance is a constant time operation, and pairing the
clients and base stations gives us O($nk$). Then, we have proved in class that Ford-Fulkerson runs in O($|f|n$) time, where $|f|$ is the size
of the max flow. As such, since the max flow in this case is $n$, this contributes O($n(e+n)$), where $e$ is the number of edges in the graph.
Therefore, combining these two we see that our overall runtime of O($(k+n+e)$)

\section{Patient-Hospital Matching}
\subsection{Problem}
After a natural disaster, $n$ people need to be sent to $k$ hospitals. A person can only go to a hospital if they are within 30 mins of the
location. Hospitals want to limit the number of people to any given hospital to $\lceil\frac{n}{k}\rceil$. Is it possible to find a
distribution such that this constraint is matched?
\subsection{Algorithm}
This problem can be modeled as a network flow and solved using Ford-Fulkerson
\begin{lstlisting}
1. Add all k hospitals and all n people to a graph. Add a virtual source S and a virtual sink T
2. For each hospital h: for each person p: if distance(h, p) < 30 mins, add a directed edge of weight 1 from p to h
3. For each hospital h add an edge from h to T with weight ceil(n/k)
4. For each person p add an edge from S to p with infinite weight
5. Run Ford-Fulkerson on the resulting graph
6. If the result of Ford-Fulkerson = n, the patients can be successfully distributed, otherwise they cannot
\end{lstlisting}
\subsection{Proof}
To prove this, we will show that the the resultant network accurately represents the situation, and that solving the max-flow problem gives
a correct solution. First, we see that this is an accurate model. By creating edges between only individuals and hospitals 30 minutes away or
less, we show in the resultant graph it is impossible for a patient to be directed to a hospital more than 30 mins away, thereby holding that
constraint. Second, by limiting output of each hospital to $T$ at $\lceil\frac{n}{k}\rceil$, by the principle of equal input and output flow,
there can be no more than $\lceil\frac{n}{k}\rceil$ incoming patients to a given hospital. Second, running Ford-Fulkerson guarantees the
production of the max-flow value for a given graph. If this value is equal to $n$, it must be the case that $n$ flow passes through the graph.
Since the only source of flow to T is through the $n$ patients, by way of how they are the only nodes connected to the source, it must be
true that 1 unit of flow is passing through each patient, meaning all patients were successfully matched. Conversely, with the given constraints
it must be true that there was at least 1 patient which was not matched with a hospital, and such a matching was therefore impossible.
\subsection{Runtime}
The runtime of this algorithm is the runtime to build the graph plus the runtime of Ford-Fulkerson. To build the graph, we must iterate over
each hospital, and for each of these, must iterate over all patients, giving us a value of O($nk$). Ford-Fulkerson runs in O($|f|(e+n)$) for a
max flow, where $|f|$ is the size of the max-flow, and $n+e$ is the number of edges and nodes in the graph. In this instance, the max-flow is
$n$, since each person contributes 1 flow, so our overall runtime if O($(k+n+e)n$) if we perform some distribution.

\section{Alternating Number Sequence}
\subsection{Problem}
Given a sequence of $n$ digits, find a longest possible subsequence of alternating numbers. Alternating numbers are defined as a sequence where
each number is either greater than or less than its two neighbors, eg (8, 2, 4, 2)
\subsection{Algorithm}
Again, this problem in a brute force manner is an O($2^n$) beast, and therefore we need dynamic programming to bring it down to polynomial time.
First, we will look at our two parameters being starting and ending position in the original sequence, and find a subsequence between these
which is maximal.
\begin{lstlisting}
1. Create an Nx2 grid with starting position on the columns and whether the last values was "up" or "down" as the rows
2. Fill the position (0, 0) with the last_value = S[0], and length = 1
3. Fill the position (0, 1) with last_value = S[0] and length = 1
4. For each further position k, set (k, 0) = (k-1, 1) + 1 if k < last_value(k-1, 1)
5. For each further position k, set (k, 1) = (k-1, 0) + 1 if k > last_value(k-1, 0)
6. Return the max of (n, 0) and (n, 1)
\end{lstlisting}
\subsection{Proof}
We prove this algorithm by looking at the exhaustive nature of its search space. We see that for each additional element from the input sequence,
we look at all cases (that it is either an up value from the previous value, or a down from the previous value, or it is not included). 
By ensuring that all of these cases are examined, we prove that we exhaustivly search for each possible value $k$, once we have reached an input
size of $n$, the value we find will indeed be maximal
\subsection{Runtime}
For the runtime, we observe that each position on our grid is computed just once, and each of these computations requires constant time. Thereofore,
given the grid size of O($n$), we see that our overall runtime is also O($n$) where $n$ is the number of input elements.
\end{document}