\documentclass[titlepage]{article}
\usepackage{amsmath}
\title{CS 180 Homework 1}
\author{Robert Geil \\
University of California, Los Angeles
}
\numberwithin{equation}{subsection}
\begin{document}
\maketitle

\section{Problem 1}

\section{Problem 2}
\section{}
\section{}
\section{Problem 5}
\subsection{Prove (by induction) that sum of 
the first n integers (1 + 2 + .... + \textit{n}) is \textit{n}(\textit{n} + 1)/2 }
\subsubsection{Base Case}
\begin{equation}
    n = 1
\end{equation}
so the original equation, when substituting \textit{n} trivially becomes
\begin{equation}
    1 = 1(1 + 1) / 2
\end{equation}
by simple evaluation of this equation in the case \textit{n} = 1, we can assure ourselves
that the equality holds. Therefore, we prove the base case

\subsubsection{Kth Case}
Assume that
\begin{equation}
    1 + 2 + ... + k = k(k + 1)/ 2
\end{equation}
We will then prove
\begin{equation}
    1 + 2 + ... + k + (k+1) = (k+1)((k+1)+1)/2
\end{equation}
\begin{equation}
    ... = (k+1)(k+2)/2
\end{equation}
Distributing the \textit{k+1} gives
\begin{equation}
    ... = (k(k+1) + 2(k+1))/2
\end{equation}
\begin{equation}
    ... = k(k+1)/2 + 2(k+1)/2
\end{equation}
\begin{equation}
    1 + 2 + ... + k + (k+1) = k(k+1)/2 + (k+1)
\end{equation}
Subtracting k+1 from both sides gives us the equation
\begin{equation}
    1 + 2 + ... + k = k(k+1)/2
\end{equation}
which by our earlier assumption is true, therefore we prove
that given the \textit{k}th case is true, the \textit{k+1}st case also holds

\subsection{What is 1\textsuperscript{3} +2\textsuperscript{3} + 3\textsuperscript{3} +...+ n\textsuperscript{3} = ?? Prove your answer by induction.}
\subsubsection{Base Case}
In the case that \textit{n = 1}, we can trivially see that the result is 1\textsuperscript{3} or 1
\end{document}