\documentclass[titlepage]{article}
\usepackage{amsmath}
\usepackage{enumerate}
\usepackage{listings}
\title{CS 180 Homework 3}
\author{Robert Geil \\
University of California, Los Angeles
}
\lstset{frame=tb,
  showstringspaces=false,
  columns=flexible,
  basicstyle={\small\ttfamily},
  breaklines=true,
  tabsize=2
}
\numberwithin{equation}{subsection}
\begin{document}
\maketitle
\section{Shortest Paths}
\subsection{Problem}
Given a graph $G$, and two nodes $v$ and $w$, find the number of
shortest paths between the two nodes.
\subsection{Algorithm}
This algorithm is very similar to pure BFS, with
the main modification being a tracking of the number of
nodes used to reach each level of the BFS tree
\begin{lstlisting}
give each node in the tree a counter set to 0, a boolean "visited" set to false and a boolean "queued" set to false
create a queue Q
set v's counter to 1, and push v into Q
while Q is not empty:
    pop from Q into node n
    for each neighboring node e adjacent to n:
        if e hasn't been visited:
            add n's counter to e's counter
        if e hasn't been "queued":
            push e into Q
            mark e as "queued"
    mark n as visited
the counter value at node w is the number of shortest paths from v to w
\end{lstlisting}
\subsection{Proof}
By the properties of Breadth First Search, the route generated from a
BFS from starting node k to any other node will be the shortest
possible path. As such, we are guaranteed that the path found from
$v$ to $w$ is the shortest possible. In addition, our algorithm
\textit{trickles down} the number of routes at each level. If our
graph has a node $n$ on level $k$ with edges to two arbitrary nodes on
level $k-1$, the two nodes will each visit the $n$. Both of
them will add their \textit{counter} to the $n$. As such,
this means the counter for the node at level $k$ can be reached
by the sum of the number of ways its two parents can be reached.
Either one (but not both) of the nodes at level $k-1$ will 
enqueue the node at level $k$. As such, we prove that the counter for
node $n$ will be representative of the sum of the nodes at a higher
level which can reach $n$. Since these levels, created implicitly by BFS,
are shortest paths, we prove that the algorithm will find the
number of shortest paths to any node in the graph $G$.
\subsection{Runtime}
Since each node is enqueued and visited only 1 time, and each edge
is only traversed once, this algorithm has the same time complexity as
pure BFS, O($n+m$). In addition, since a counter needs to be stored
at each node, there is a space complexity of O($n$).
\section{Depth-First Breadth-First Equivalence}
\subsection{Problem}
Given a graph $G$, choose an node $u$ and using depth-first search,
build a tree $T$ rooted at $u$. Using breadth-first search, build a
tree $T'$ rooted at $u$. If $T'$ = $T$, prove that $T$ = $G$.
\subsection{Proof}
By contradiction, assume that $G$ does not equal $T$. In this case,
there must be a cycle in $G$, as a tree $T$ follows the property
of being acyclic. If there is a cycle in $G$, there is some node $n$
at a \textit{level} $k$ away from the root $u$ such that $n$ must have
an edge to another node $n'$ in $k$ as well, by the definition of a cycle
in a BFS tree. Here level indicates the distance from the root to a set of
nodes. As such, if such an edge existed, a DFS traversal, by the algorithm
of DFS would visit that edge first, as DFS recursively the first edge
of a node and then visit that node. As such, the tree created by DFS would
have the other cyclic node on a level $k+1$, but the BFS search, as it finds
the shortest path from the root to each node, would have it on level $k$.
Therefore, without loss of generality, we find a contradiction and show that
if $G$ is not equal to $T$, BFS and DFS will generate different trees.
\section{A Connected Network}
\subsection{Problem}
A group of $n$ (where $n$ is even) mobile devices form a network,
where nodes $i$ and $j$ are connected by an edge if the distance between
$i$ and $j$ is less than or equal to 500 meters. Can network connectivity
be ensured by requiring that each device is no more than 500 meters from
at least $n/2$ other devices at any time? That is, if there is a 
graph $G$ of $n$ edges, is $G$ guaranteed to be connected if each
node has a degree of at least $n/2$?
\subsection{Proof}
Yes, if each node has a degree of $n/2$, the graph is guaranteed to be
connected. By definition of degree of a vertex, an arbitrary node $k$
must be connected to $n/2$ \textit{other} nodes. Assume there is another
node $k'$ that is not connected to $k$. By the rules of the graph, 
$k'$ must also be connected to $n/2$ other nodes. By the definition of
connected, any node that is connected to both $k$ and $k'$ implies that
$k$ and $k'$ are connected, so therefore the sets $S$ and $S'$ (all the
nodes connected to $k$ and $k'$ respectively) must be disjoint. In
addition, $k$ is not in $S$ (since a node cannot be connected to itself)
and $k$ is not in $S'$, as that would imply it is connected to $k'$.
Similarly $k'$ is not in $S$ or $S'$. By graph rules, the cardinality of 
$S$ and $S'$ must each be at least $n/2$. Therefore, the cardinality of
the union of $S$, $S'$, $k$ and $k'$ must be at least $n/2 + n/2 + 1 + 1$.
or $n + 2$. Because the graph contains only $n$ nodes, this forms a 
contradiction, and we prove that any two arbitrary nodes in a graph $G$
each with degree $n/2$ \textit{must} be connected.
\end{document}
