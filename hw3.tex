\documentclass[titlepage]{article}
\usepackage{amsmath}
\usepackage{enumerate}
\usepackage{listings}
\title{CS 180 Homework 3}
\author{Robert Geil \\
University of California, Los Angeles
}
\lstset{frame=tb,
  showstringspaces=false,
  columns=flexible,
  basicstyle={\small\ttfamily},
  breaklines=true,
  tabsize=2
}
\numberwithin{equation}{subsection}
\begin{document}
\maketitle
\section{Shortest Paths}
\subsection{Problem}
Given a graph $G$, and two nodes $v$ and $w$, find the number of
shortest paths between the two nodes.
\subsection{Algorithm}
This algorithm is very similar to pure BFS, with
the main modification being a tracking of the number of
nodes used to reach each level of the BFS tree
\begin{lstlisting}
give each node in the tree a counter set to 0, a boolean "visited" set to false and a boolean "queued" set to false
create a queue Q
set v's counter to 1, and push v into Q
while Q is not empty:
    pop from Q into node n
    for each neighboring node e adjacent to n:
        if e hasn't been visited:
            add n's counter to e's counter
        if e hasn't been "queued":
            push e into Q
            mark e as "queued"
    mark n as visited
the counter value at node w is the number of shortest paths from v to w
\end{lstlisting}
\subsection{Proof}
By the properties of Breadth First Search, the route generated from a
BFS from starting node k to any other node will be the shortest
possible path. As such, we are guaranteed that the path found from
$v$ to $w$ is the shortest possible. In addition, our algorithm
\textit{trickles down} the number of routes at each level. If our
graph has a node $n$ on level $k$ with edges to two arbitrary nodes on
level $k-1$, the two nodes will each visit the $n$. Both of
them will add their \textit{counter} to the $n$. As such,
this means the counter for the node at level $k$ can be reached
by the sum of the number of ways its two parents can be reached.
Either one (but not both) of the nodes at level $k-1$ will 
enqueue the node at level $k$. As such, we prove that the counter for
node $n$ will be representative of the sum of the nodes at a higher
level which can reach $n$. Since these levels, created implicitly by BFS,
are shortest paths, we prove that the algorithm will find the
number of shortest paths to any node in the graph $G$.
\subsection{Runtime}
Since each node is enqueued and visited only 1 time, and each edge
is only traversed once, this algorithm has the same time complexity as
pure BFS, O($n+m$). In addition, since a counter needs to be stored
at each node, there is a space complexity of O($n$).
\section{Depth-First Breadth-First Equivalence}
\subsection{Problem}
Given a graph $G$, choose an node $u$ and using depth-first search,
build a tree $T$ rooted at $u$. Using breadth-first search, build a
tree $T'$ rooted at $u$. If $T'$ = $T$, prove that $T$ = $G$.
\subsection{Proof}
By contradiction, assume that $G$ does not equal $T$. In this case,
there must be a cycle in $G$, as a tree $T$ follows the property
of being acyclic. If there is a cycle in $G$, there is some node $n$
at a \textit{level} $k$ away from the root $u$ such that $n$ must have
an edge to another node $n'$ in $k$ as well, by the definition of a cycle
in a BFS tree. Here level indicates the distance from the root to a set of
nodes. As such, if such an edge existed, a DFS traversal, by the algorithm
of DFS would visit that edge first, as DFS recursively the first edge
of a node and then visit that node. As such, the tree created by DFS would
have the other cyclic node on a level $k+1$, but the BFS search, as it finds
the shortest path from the root to each node, would have it on level $k$.
Therefore, without loss of generality, we find a contradiction and show that
if $G$ is not equal to $T$, BFS and DFS will generate different trees.
\section{A Connected Network}
\subsection{Problem}
A group of $n$ (where $n$ is even) mobile devices form a network,
where nodes $i$ and $j$ are connected by an edge if the distance between
$i$ and $j$ is less than or equal to 500 meters. Can network connectivity
be ensured by requiring that each device is no more than 500 meters from
at least $n/2$ other devices at any time? That is, if there is a 
graph $G$ of $n$ edges, is $G$ guaranteed to be connected if each
node has a degree of at least $n/2$?
\subsection{Proof}
Yes, if each node has a degree of $n/2$, the graph is guaranteed to be
connected. By definition of degree of a vertex, an arbitrary node $k$
must be connected to $n/2$ \textit{other} nodes. Assume there is another
node $k'$ that is not connected to $k$. By the rules of the graph, 
$k'$ must also be connected to $n/2$ other nodes. By the definition of
connected, any node that is connected to both $k$ and $k'$ implies that
$k$ and $k'$ are connected, so therefore the sets $S$ and $S'$ (all the
nodes connected to $k$ and $k'$ respectively) must be disjoint. In
addition, $k$ is not in $S$ (since a node cannot be connected to itself)
and $k$ is not in $S'$, as that would imply it is connected to $k'$.
Similarly $k'$ is not in $S$ or $S'$. By graph rules, the cardinality of 
$S$ and $S'$ must each be at least $n/2$. Therefore, the cardinality of
the union of $S$, $S'$, $k$ and $k'$ must be at least $n/2 + n/2 + 1 + 1$.
or $n + 2$. Because the graph contains only $n$ nodes, this forms a 
contradiction, and we prove that any two arbitrary nodes in a graph $G$
each with degree $n/2$ \textit{must} be connected.
\section{Optimal Shipping}
\subsection{Problem}
A shipping company is sending one truck between two locations each day.
Based on company policies, only 1 truck may be used at a time and packages
must be sent out in the order they are received. The company is 
currently using a greedy algorithm to pack the truck, sending the most 
recently delivered packages out until the truck cannot accomadate
the next package, and then waiting for the next truck. Prove that this
is an optimal solution
\subsection{Proof}
Assume there is some optimal solution $O$ which sends the fewest possible
trucks to transport the packages. We can see that the existing solution $S$
and the optimal solution $O$ must load packages identically for some
package number $i$, such that $i\geq0$. This can be trivially proved, since
at time 0, both our solution and the optimal solution have packed no packages.
For sending the next package, consider the two possibilities. First, consider
the case that the $i+1$st package is too large for the current truck. In this case,
both our solution and the optimal solution must wait for the next truck,
based on the rules of packing, and therefore are equivelent. Second, the
case that the $i+1$st case can fit on the truck, but may choose to be delayed.
In the implementation of $S$, the package is greedily sent on the current truck.
The optimal solution minimizes the number of individual trucks sent. 
In order to do this, we assume there is some minimum required number of
trucks to transport all the remaining packages. For example, if the capacity
of each truck is 10 and there are 20 packages of weight 1, there must be at least
2 truckloads sent out. As such, the optimal solution sends a minimum number
of trucks out. However, we see that our solution $S$, if a package can fit on
a truck will ship it. As such, after the $i+1$st package, our solution
will always have at most the same number of packages remaining to be sent
as the optimal solution, and will not have sent an additional truck to
send the $i+1$st package. As such, we prove by induction that our solution $S$
is equivelent to the optimal solution $O$ up to a point $i$, and that for
every package after that point, our solution performs at least as well
as the optimal solution.
\section{Triathalon Ordering}
\subsection{Problem}
You are running a triathalon, with swimming, biking and running components,
and a group of individuals competing. The contestants are sent out in
a staggered fashion, because only 1 person may swim at a time, but any
number of individuals may run/cycle concurrently. Given an \textit{estimated}
time for running, swimming and cycling, how can you order individuals such that
the last individual finishes as soon as possible?
\subsection{Algorithm}
\begin{lstlisting}
sort the contestants based on the sum of running and biking time
while there are contestants remaining:
    send the next contestant k into the pool
    wait for k to finish swimming
\end{lstlisting}
\subsection{Proof}
The algorithm relies on the fact that concurrancy is permitted for
running and cycling, but not for swimming. Because only 1 person is
permitted in the pool at a time, there exists a minimum amount of
time that the contest must run for, namely the total time of all
swimmers, as there cannot be overlapping. As such, the criteria to
optimize is the amount of time that contestants spend running and
biking. By selecting the individual who takes the most time to bike/run
and starting them first, it is guaranteed that anyone who swims after them
will take less time biking/running. To prove this, consider an optimal solution $O$,
and our solution $S$. We see there is some initial ordering up to step $i \geq 0$
where $S$ and $O$ agree. At the $i+1$st step, $S$ chooses the individual who
will take longest to complete the running/walking. By choosing this, we can
see that this individual will finish swimming faster and begin running before
another candidate. Therefore the minimum remaining time is $S+(R+B)$ where 
$S$ is the time to swim, $R$ is the time to run and $B$ is the time to bike.
This is at least as good as the optimal solution, as the remaining candidates will
have more overlapped time to run and bike.
\subsection{Runtime}
The only significant step of this algorithm is performing a sort based on
combined running and swimming time. As sorting can be done in O($n\log(n)$) time,
and following the sort is just a linear time pass through the list, the
overall time complexity of this algorithm is O($n\log(n)$)
\section{Longest Paths}
\subsection{Problem}
Given a graph $G$, what is the longest path that can be made between
any two nodes, if edges can only be traversed once?
\subsection{Directed Graph}
\subsubsection{Algorithm}
This problem cannot be solved in polynomial time,
as any possible subset of the paths could be potentially
the longest path, especially as cycles must be considered.
\begin{lstlisting}
enumerate all possible unique paths
find the longest of the enumerated paths
\end{lstlisting}
\subsubsection{Proof}
By observing every possible path, it is guaranteed that the
longest path must be in this set. Therefore, this algorithm is
guaranteed to find the longest possible path
\subsubsection{Runtime}
The runtime of this algorithm is O($n!$). This problem is akin to
other NP-Complete problems, like the Traveling Salesman Problem. In
the case that the graph is completely connected, this problem is the same
as the TSP, except that the optimization criteria is to find the longest,
not shortest path passing through all nodes. As an NP-Complete problem,
there is no known polynomial time solution.
\subsection{Directed, Acyclic Graph}
\subsubsection{Algorithm}
To find a solution, one can perform an algorithm similar to a reverse topological ordering\\
\begin{minipage}{\linewidth}
\begin{lstlisting}
create a set S
mark each node in G with a distance of 0
mark each node in G with a 'next' of null
fill S with nodes that are not a source of an edge
while G is not empty:
    for each node k in S:
        for each node o incoming to k:
            if o's distance is less than k's distance + 1:
                set o's distance to k's distance + 1
                set o's next to k
        remove k from S
        remove k from G
    repopulate S with all nodes in the updated G with no incoming edges
search all nodes to find the largest distance
the node with the max distance is the longest path, and the path can be calculated by following the 'next' values
\end{lstlisting}
\end{minipage}
\subsubsection{Proof}
The algorithm follows somewhat to a topological ordering. We will
prove the accuracy by induction. We see for the first step selecting
all nodes with no outgoing edges, there is a guarantee that the nodes
which go to these terminal nodes have a path length of at least 1, but
it cannot be determined if there is a longer path yet. For the nth step,
we see that each time a node with no outgoing edges is examined, that path
from a previous node to the terminal must be at least the length
of the terminal path plus 1. Using a comparison, and only updating
the path if the distance from a terminal back to a non-terminal
parent is greater than the existing distance to the parent,
we maximize the length of the path. By this induction, we prove that the
algorithm will find the longest path in a Directed Acyclic Graph.
\subsubsection{Runtime}
This algorithm, in performing similar to a topological in that it examines
each edge once and removes nodes once all their edges have been examined
gives us a runtime of O($n+m$). In addition, searching for the maximum value
is only an O($n$) operation, maintaining our time complexity of O($n+m$).
\end{document}
